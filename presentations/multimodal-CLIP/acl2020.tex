%
% File acl2020.tex
%
%% Based on the style files for ACL 2020, which were
%% Based on the style files for ACL 2018, NAACL 2018/19, which were
%% Based on the style files for ACL-2015, with some improvements
%%  taken from the NAACL-2016 style
%% Based on the style files for ACL-2014, which were, in turn,
%% based on ACL-2013, ACL-2012, ACL-2011, ACL-2010, ACL-IJCNLP-2009,
%% EACL-2009, IJCNLP-2008...
%% Based on the style files for EACL 2006 by 
%%e.agirre@ehu.es or Sergi.Balari@uab.es
%% and that of ACL 08 by Joakim Nivre and Noah Smith

\documentclass[11pt,a4paper]{article}
\usepackage[hyperref]{acl2020}
\usepackage{times}
\usepackage{latexsym}
\usepackage{graphicx}
\usepackage{hyperref}
\renewcommand{\UrlFont}{\ttfamily\small}

\usepackage{microtype}

\aclfinalcopy % Uncomment this line for the final submission #TODO 
%\def\aclpaperid{***} %  Enter the acl Paper ID here


\newcommand\BibTeX{B\textsc{ib}\TeX}

\title{Multimodal Embeddings} 

\author{Jan Kels \\
  Heinrich Heine Universität \\
  Düsseldorf, Germany \\
  \texttt{Jan.Kels@hhu.de} \\\and
  Omar Hassan\\
  Heinrich Heine Universität \\
  Düsseldorf, Germany \\
  \texttt{Omar.Hassan@hhu.de}
  }

\date{24.05.2023}

\begin{document}
\maketitle
\begin{abstract} %do we even need this part?
  we present the concept of multimodal embedding and highlight the CLIP model architecture.
\end{abstract}

\section{Introduction} %seem repetitive
multimodal embeddings combine different modalities such as text(language), audio and images.

\section{CLIP} %maybe not the first section, text in progress
The CLIP \cite{radford2021learning} architecutre uses contrastive loss to learn a common embedding space for image, text pairs.
It's based on consine similarity and can be used with a variety of text encoder as well as image encoders. 
The main task for CLIP models is zero shot image embeddings, where you construct a classifier by crafting a prompt.
A single pretrained model can outperform task specific fine tuned models on various benchmarks. Not all.
The blog post provides a great overview.

\subsection{Applications} %I will look for a reference on those saliency maps
CLIP is the backbone for many text to image generation models such as DALL-E 2 \cite{ramesh2022hierarchical} ("unclip").
Academic use such in the Visual Word Sense Disambiguation \cite{raganato-etal-2023-semeval} task.

\subsection{following research} %better title needed
BLIP, BLIP2, cross attention from prompt2prompt paper, dino robustness?

\bibliography{acl2020}
\bibliographystyle{acl_natbib}

\appendix
\end{document}
